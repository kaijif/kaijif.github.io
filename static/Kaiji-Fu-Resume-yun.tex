\documentclass[letterpaper,11pt]{article}

\usepackage{latexsym}
\usepackage[empty]{fullpage}
\usepackage{titlesec}
\usepackage{marvosym}
\usepackage[usenames,dvipsnames]{color}
\usepackage{verbatim}
\usepackage{enumitem}
\usepackage[hidelinks]{hyperref}
\usepackage{fancyhdr}
\usepackage[english]{babel}
\usepackage{tabularx}
\input{glyphtounicode}


%----------FONT OPTIONS----------
% sans-serif
% \usepackage[sfdefault]{FiraSans}
% \usepackage[sfdefault]{roboto}
% \usepackage[sfdefault]{noto-sans}
% \usepackage[default]{sourcesanspro}

% serif
\usepackage{CormorantGaramond}
\usepackage{charter}


\pagestyle{fancy}
\fancyhf{} % clear all header and footer fields
\fancyfoot{}
\renewcommand{\headrulewidth}{0pt}
\renewcommand{\footrulewidth}{0pt}

% Adjust margins
\addtolength{\oddsidemargin}{-0.5in}
\addtolength{\evensidemargin}{-0.5in}
\addtolength{\textwidth}{1in}
\addtolength{\topmargin}{-.5in}
\addtolength{\textheight}{1.0in}

\urlstyle{same}

\raggedbottom
\raggedright
\setlength{\tabcolsep}{0in}

% Sections formatting
\titleformat{\section}{
  \vspace{-4pt}\scshape\raggedright\large
}{}{0em}{}[\color{black}\titlerule \vspace{-5pt}]

% Ensure that generate pdf is machine readable/ATS parsable
\pdfgentounicode=1

%-------------------------
% Custom commands
\newcommand{\resumeItem}[1]{
  \item\small{
    {#1 \vspace{-2pt}}}
}
\newcommand{\resumeSubheading}[2]{
  \vspace{-2pt}\item
    \begin{tabular*}{0.97\textwidth}[t]{l@{\extracolsep{\fill}}r}
      \textbf{#1} & #2 \\
    \end{tabular*}
    \vspace{-7pt}
}

\newcommand{\resumeSubSubheading}[2]{
    \item
    \begin{tabular*}{0.97\textwidth}{l@{\extracolsep{\fill}}r}
      \textit{\small#1} & \textit{\small #2} \\
    \end{tabular*}\vspace{-7pt}
}

\newcommand{\resumeProjectHeading}[2]{
    \item
    \begin{tabular*}{0.97\textwidth}{l@{\extracolsep{\fill}}r}
      #1 & #2 \\
    \end{tabular*}
    \vspace{-7pt}
}

\newcommand{\resumeSubItem}[1]{\resumeItem{#1}\vspace{-4pt}}

\renewcommand\labelitemii{$\vcenter{\hbox{\tiny$\bullet$}}$}

\newcommand{\resumeSubHeadingListStart}{\begin{itemize}[leftmargin=0.15in, label={}]}
\newcommand{\resumeSubHeadingListEnd}{\end{itemize}}
\newcommand{\resumeItemListStart}{\begin{itemize}[leftmargin=0.15in]}
\newcommand{\resumeItemListEnd}{\end{itemize}\vspace{-5pt}}

%-------------------------------------------
%%%%%%  RESUME STARTS HERE  %%%%%%%%%%%%%%%%%%%%%%%%%%%%


\begin{document}

%----------HEADING----------
\begin{center}
    \textbf{\Huge \scshape Kaiji Fu} \\ \vspace{4pt}
    \small +1 (252) 267-0412 $|$ \href{mailto:hello@kaijifu.com}{\underline{hello@kaijifu.com}} $|$ US Citizen
\end{center}


%-----------EDUCATION-----------
\section{Education}
  \resumeSubHeadingListStart
    \resumeSubheading
      {University of North Carolina at Chapel Hill}{May 2026}
      \begin{tabular*}{0.97\textwidth}[t]{l}
        \textit{\small{B.S. Computer Science and Mathematics $|$ GPA: 4.0 $|$ Carolina Scholar (full scholarship, top 1\%) $|$ Honors College (top 10\%)}} \\
      \end{tabular*}
    %   \begin{itemize}[leftmargin=0in, label={}]
    %     \small{\item{
    %      \textbf{Relevant Coursework:}{ Data Structures, Systems Programming, Algorithms \& Analysis, Discrete Mathematics} \\
    %     }}
    %  \end{itemize}
  \resumeSubHeadingListEnd
%-----------EXPERIENCE-----------
\section{Experience}
  \resumeSubHeadingListStart
    \resumeSubheading
      {\textbf{Software Engineering Intern, Full Stack} - \textit{Ember Learning}}{May 2024 -- Aug 2024}
      \resumeItemListStart
        \resumeItem{Completed a summer internship at an \textbf{AI} education platform, helping increase \textbf{MAU by 100x} and \textbf{revenue by 1000x}}
        \resumeItem{Shipped an \textbf{AI} grading feature by building \textbf{20+} responsive user interface components using \textbf{TypeScript} and \textbf{React}}
        \resumeItem{Built \textbf{AWS} backend supporting \textbf{100,000+ users} and \textbf{3M+} AI-graded questions w/ \textbf{Terraform}, generating \textbf{\$40k+ ARR}}
        \resumeItem{Implemented fine-tuning pipelines for \textbf{LLMs}, improving grading accuracy across diverse standards by more than \textbf{60\%}}
      \resumeItemListEnd
      \resumeSubheading
      {\textbf{Software Engineer, Open Source} - \textit{Mozilla}}{Dec 2023 -- Present}
      \resumeItemListStart
        \resumeItem{Contributed to Mozilla's bugbug project, a bug classification system that uses \textbf{ML} to triage \textbf{10k+} Firefox bugs/month} 
        \resumeItem{Implemented critical fixes in \textbf{Python} for type-checking issues, merging \textbf{2000+} lines of code over \textbf{20+ pull requests}}
        \resumeItem{Collaborated with core maintainers through \textbf{GitHub issues} and \textbf{code reviews} to ensure code quality and compatibility}
      \resumeItemListEnd
      \resumeSubheading
      {\textbf{Research Assistant (AI/ML)} - \textit{UNC School of Medicine}}{ Nov 2024 -- Present}
      \resumeItemListStart
        \resumeItem{Collaborated with cardiologists to develop \textbf{transformer architectures} to analyze ECGs and detect cardiac anomalies}
        \resumeItem{Used \textbf{Python} and \textbf{Tensorflow} on \textbf{high-performance Linux} to achieve \textbf{11\% higher accuracy} than the state of the art}
      \resumeItemListEnd
      \resumeSubheading
      {\textbf{Research Assistant (AI/ML)} - \textit{ECU School of Medicine}}{Sept 2022 -- Feb 2023}
      \resumeItemListStart
        \resumeItem{Researched how federated (distributed) \textbf{machine learning} enhances patient privacy when training diagnostic models}
        \resumeItem{Demonstrated that federated modeling maintains \textbf{95\%+} accuracy while obviating cross-institutional data sharing}
        \resumeItem{Won \textbf{Best Poster} at the ISS Symposium, where I presented findings to faculty, industry partners, and fellow researchers}
      \resumeItemListEnd
  \resumeSubHeadingListEnd
%-----------PROJECTS-----------
\section{Projects}
    \resumeSubHeadingListStart
      \resumeProjectHeading
          {\textbf{Nolyn} $|$ \emph{AWS/cloud, React, full-stack, C/C++, RTOS, embedded development/debugging}}{ May 2023 -- Dec 2024}
          \resumeItemListStart
            % \resumeItem{\textbf{Founded a startup} to build stop-arm cameras for school buses (automated systems that capture license plates of vehicles illegally passing stopped buses) at \textbf{100x lower cost} than competitors (\$30 vs up to \$3,000 each)}
            \resumeItem{Founded a startup that built a camera to capture license plates of vehicles illegally passing buses for \textbf{100x} lower cost}
            \resumeItem{Developed microcontoller firmware with \textbf{C/RTOS} and connected it to \textbf{AWS} for real-time image capture and analysis}
            % TODO revise
            \resumeItem{Engineered a \textbf{full-stack cloud application} with \textbf{AWS} and a \textbf{ReactJS} admin portal, providing \textbf{100\%} real-time visibility}
            \resumeItem{Deployed on \textbf{2000+} buses across \textbf{10+} school districts and secured \textbf{\$1k+} in venture capital from investors like Amazon}
          \resumeItemListEnd
      % \vspace{-4pt}
      \resumeProjectHeading
        {\textbf{Blackbeard} $|$ \emph{OpenCV, PyTorch, CUDA, robotics, embedded development, computer vision}}{Aug 2022 -- May 2023}
        \resumeItemListStart
          \resumeItem{Trained an AI object detection model with \textbf{OpenCV/PyTorch} to \textbf{4x} \textbf{self-driving} performance in a robotics competition}
          \resumeItem{Deployed the model on an \textbf{embedded Linux} coprocessor, achieving \textbf{95\%} accurate real-time detection of field elements}
          \resumeItem{Implemented \textbf{MQTT} protocol with \textbf{C++} and \textbf{Java} to connect coprocessor and robot controller for reliable data transfer}
        \resumeItemListEnd
        % \vspace{-4pt}
        \resumeProjectHeading
          {\textbf{72o} $|$ \emph{Python, Numpy, Pandas, Machine Learning, Game Theory}}{Feb 2025}
          \resumeItemListStart
            \resumeItem{Collaborated with team of 4 to build a pokerbot with \textbf{Python} that placed \textbf{1st/112} in the UNC Pokerbots competition}
            \resumeItem{Applied \textbf{counterfactual regret minimization (CFR)} algorithms to develop game-theory optimal betting strategie}
            \resumeItem{Engineered an opponent modeling system capable of adapting to villain's play patterns, increasing winrate by \textbf{32\%}}
            \resumeItem{Leveraged \textbf{multi-threading} to parallelize decision-making, decreasing latency by \textbf{3x} and avoiding disqualification}
          \resumeItemListEnd
          \resumeProjectHeading
          {\textbf{Loggerhead} $|$ \emph{Swift, AWS, PostgreSQL, iOS development, RESTful APIs}}{Jan. 2021 -- Feb. 2024}
            \resumeItemListStart
              \resumeItem{Developed a \textbf{full-stack} \textbf{iOS application} in \textbf{Swift} to track and analyze tennis practice sessions with ball machines}
              \resumeItem{Designed and implemented a \textbf{RESTful API} using \textbf{AWS Lambda} and \textbf{API Gateway} to store and retrieve user data}
              \resumeItem{Created a robust data model with \textbf{PostgreSQL} for tracking practice metrics, ball machine settings, and analytics}
              \resumeItem{Implemented progress visualization with \textbf{SwiftUI}, helping users track improvement through data-driven insights}
          \resumeItemListEnd
    \resumeSubHeadingListEnd
%-----------TECHNICAL SKILLS-----------
\section{Technical Skills}
 \begin{itemize}[leftmargin=0.15in, label={}]
    \small{\item{
     \textbf{Languages}{: Python, Bash, JavaScript, TypeScript, HTML, CSS, C, C++, Java, Swift} \\
     \textbf{Frameworks}{: React, PyTorch, TensorFlow, Pandas, TailwindCSS, Svelte, Angular, Object-Oriented Programming} \\
     \textbf{Developer Tools}{: Terraform, Git, GitHub, AWS, IAM, Jira, Docker, CI/CD, Linux, Kubernetes, JUnit} \\
     \textbf{Domain Knowledge}{: Cloud Computing, RESTful APIs, IaC,  AI, LLMs, Embedded Systems, iOS, Open Source}
    }}
 \end{itemize}

%-------------------------------------------
\end{document}